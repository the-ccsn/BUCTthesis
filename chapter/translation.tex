%% 翻译--translation.tex
\begin{translation}
    Life without batteries is inconceivable. 
    Stored energy has become an integral part of our everyday lives. 
    Without this over 100-year-old technology, 
    the success story of laptops, cell phones, 
    and tablets would not have been possible. 
    Although there are many ways of storing power, 
    there is only one system that enables the functions 
    that meet consumers' expectations 
    of a storage medium --- the rechargeable battery. 
    A battery that can be discharged and charged at the push of a button. 
    Strictly speaking, the battery is not a storage system for electric power 
    but an electrochemical energy converter. 
    And in recent decades its development has followed many convoluted paths.

    没有电池的生活是无法想象的,
    储蓄能源已经成为了我们日常生活中不可分割的一部分。
    如果没有了这项已逾百年的技术,
    那么电灯、电话和平板电脑的成功将会被改写。
    尽管储存能量的方式有许多,但是仅有一种方法,
    使得满足消费者对存储介质的期望成为可能,即可充电电池。
    按下按钮,电池就能充放电。
    严格地讲,电池并不是电能的存储系统,
    而是电能和化学能之间的转化器。
    在近几十年以来,它沿着复杂的道路发展着。
    
    The history of the battery, both as a primary and secondary element, 
    has not yet been fully elucidated today. 
    We know that the voltaic pile was introduced by A.~Volta (1745-1827) around 1800. 
    Some 65 years later, around 1866, G.~Leclanché (1839-1882) was 
    awarded a patent for a primary element, the so-called Leclanché element. 
    The element consisted of a zinc anode, a graphite cathode, 
    and an electrolyte made of ammonium chloride. The cathode had a manganese dioxide 
    coating on the boundary surface with the electrolyte. 
    C.~Gassner (1855-1942) further developed this system, and in 1901 
    P.~Schmidt (1868-1948) succeeded in inventing the first galvanic dry element 
    based on zinc and carbon.
    
    电池,作为一次电池和二次电池,其历史至今都尚未完全阐明。
    我们知道,约 1800 年,A.~Volta (1745-1827) 发明了伏打电堆;
    然后约 65 年后,也就是在 1866 年附近,G.~Leclanché (1839-1882)
    被授予一项和电池有关的专利,即所谓“勒克朗谢电池”。
    这种电池使用锌作正极、石墨作负极,和以氯化铵制备的电解质;
    负极在与电解质的界面上有着二氧化锰涂层。
    不久,C.~Gassner (1855-1942) 改进了这种体系,接着
    在1901年,P.~Schmidt (1868-1948) 成功地用锌和石墨发明了第一节干电池。
    
\end{translation}