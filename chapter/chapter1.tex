%% 第一章--chapter1.tex
\chapter{模板使用之前}
这是一份北京化工大学本科生毕业论文模板的使用指南。
这份文档亦由该模板编译而成。
本指南的目的在于帮助北化本科毕业生掌握此模板的使用方法,从而写出符合北化《本科生毕业设计(论文)撰写规范》要求的毕业论文。

本模板的内容范围仅限于毕业设计(论文)文本,且不含论文封面。开题报告、任务书、中期检查表、评阅意见表及其说明、评分手册与优秀毕业设计(论文)简介等非毕业设计(论文)文本部分不在其中。

	\section{概述}
	本写作模板基于\CTeX{}的book 文类,所以 book 的选项对于本模板也是有效的,且默认编码为 UTF-8。
	本模板已在Windows\ 10(64bit)操作系统下,\TeX{}Live 2019,使用 XeLaTeX 编译通过,
	至于Mac和Linux系统不能保证完美运行。
	由于此模板的\emph{参考文献}使用\BibTeX{},所以完整的编译链为:
	XeLaTeX $\rightarrow$ BibTeX{}$\rightarrow$ XeLaTeX{}$\rightarrow$ XeLaTeX{}。
		\subsection{开源说明}
			本项目开源于GitHub\footnote{\url{https://github.com/Miracle0565/BUCTthesis}},若要反馈bug(s)请于此提交Issues。本模板不做任何商业用途。
		\subsection{模板组成}
		在表 \ref{tab:mainfile} 中罗列了本模板所包含的文件。
		\begin{table}%[H]
			\centering
			\caption{主要文件}
			\label{tab:mainfile}
			\begin{tabular}{ll}
				\hline 
				文件(夹)名 & 简述 \\ 
				\hline 
				\textsf{buctthesis.tex} & 主文件 \\
				\textsf{buctthesis.sty} & \LaTeX{}宏包文件\\
				\textsf{buctthesis.pdf} & 使用指南,即您正在阅读的这个文件\\
				\textsf{gbt7714-2005.bst} & \BibTeX{}用到的参考文献格式模板\\
				\textsf{thesisbib.bib}		& \BibTeX{}参考文献数据库文件\\
				\textsf{chapter/} & 各个“章”的源文件路径\\
				\textsf{code/} & 源代码的路径\\
				\textsf{figure/} & 图片的路径\\
				\textsf{scanPDF/} & 存放已扫描的封面等文件的路径\\
				\hline 
			\end{tabular} 
		\end{table}
		以下是各个文件(夹)的详细介绍:
		\subsubsection{\textsf{buctthesis.tex}}主文件。
		内含少量注释,一般您需要做的有:
		删改中英文标题的名称、增添正文的章节、增添附录的章节。
		具体方式请参考第 \ref{chap:CodeIntro} 章。
		编译此文件以形成PDF。
		\subsubsection{\textsf{buctthesis.sty}}
		\LaTeX{}宏包文件,起到格式控制作用。
		该文档已经做了较为充分的注释,若您觉得依此宏包编译出的文档有不美观甚至有错误的地方,
		可以在其中对应处做些许修改。
		\subsubsection{\textsf{gbt7714-2005.bst}}
		参考文献格式控制文件,使得\emph{参考文献}部分符合规范。此文件包含其制作者信息。
		\subsubsection{\textsf{chapter/}、\textsf{figure/}和\textsf{code/}}
		这三个文件夹分别存放对应的文件,在第 \ref{chap:CodeIntro} 章会讲述如何将各个章节、图片和源代码等插入至文章的相应位置中。

	\section{一些编辑器的初始配置}
		开始之前,最好能使用最新的发行版\LaTeX{},使用旧版可能会有潜在的问题;同时在查看环境变量(此电脑$\rightarrow$ 属性$\rightarrow$ 高级系统设置$\rightarrow$ 环境变量$\rightarrow$ 系统变量\,中),请确保\texttt{C:\cmd{Windows}\cmd{system32}}\footnote{盘符 C 表示系统所在盘}位于其中。

		以下仅介绍 Win10 系统下一些常用的编辑器的基本配置。
		\subsection{\href{http://texstudio.sourceforge.net}{\TeX{} Studio}}
			编译时选择\XeLaTeX{}:
			\emph{选项(O)} $\rightarrow$ \emph{设置 TeX Studio} $\rightarrow$ \emph{构建} $\rightarrow$ \emph{默认编译器} 中选择\texttt{XeLaTeX},编译时会根据实际情况判断是否执行一个完整的编译链。

		\subsection{\href{https://code.visualstudio.com}{Visual Studio Code}}
		初次使用做如下配置:
		\begin{itemize}
			\item 将\TeX{}Live下的 \texttt{\dots \cmd{textlive}\cmd{2019}\cmd{bin}\cmd{win32}} 加入至系统环境变量之中,这是您所安装 \TeX{}Live 的路径;
			\item 在\emph{应用商店}(Extensions)中下载扩展:\textsf{LaTeX Workshop},您也可以在\href{https://github.com/James-Yu/LaTeX-Workshop}{这里}查看关于此扩展的详细说明;
			\item 在\emph{设置}(Settings)中的搜索框输入 latex,单击搜索显示的第一项\emph{在setting.json中编辑},并添加\textsf{code/setting.json}的代码。
		\end{itemize}
		
		编译时,在工具栏中\texttt{COMMANDS}内\texttt{Build LaTeX project}选择所需的编译方式。已配置默认编译为 XeLaTeX,且可手动选择完整的编译链。

		若您偏好于其它的一些编辑器如 \href{http://www.lyx.org/}{LyX},在线编译器如 \href{https://www.overleaf.com/}{Overleaf},或是 \TeX{}Live 自带的 TeXworks, 因作者能力有限,无法一一介绍其配置及操作,还请您自己在网上寻求更多帮助。